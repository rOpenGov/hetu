% !TeX root = RJwrapper.tex
\title{Validating and Extracting Information from National Identification Numbers in R: The Case of Finland and Sweden}
\author{by Pyry Kantanen, Måns Magnusson, Erik Bülow,  Jussi Paananen, Leo Lahti} % Add affiliations?

\maketitle

\abstract{
National identification numbers (NIN) are a widely used method for uniquely identifying individuals and companies in Finland, Sweden, and many other countries. Their widespread use in both public sector registers and private sector customer data systems prompts the need for openly available methods and tools for NIN analysis and validation. The \CRANpkg{hetu} and \CRANpkg{sweidnumbr} R packages provide functions for extracting information from Finnish and Swedish NINs. The packages can also generate temporary but structurally valid random numbers for testing purposes. Finnish and Swedish NINs contain information about a person's birth date and sex, an individual number and a control character that helps detect most common input errors and structural irregularities. Our packages contain comprehensive documentation and examples to make usage of the packages easy for domain experts with non-technical backgrounds. While the introduced tools are relevant mainly in the Finnish and Swedish context, this work contributes to developing similar tools for other countries. From a wider perspective, these packages contribute to the growing toolkit of standardized methods for computational social science research, epidemiology, demographic studies and other register-based inquiries.}

\section{Introduction}

Technical systems for identifying people, organizations, places and other objects are an important but often overlooked aspect of governance and management tools in modern societies \citep{dodge2005}. Universal and persistent identification numbering systems for natural persons are vital for facilitating research activities that combine data from different sources, for example in the fields of epidemiology, population studies and social research \citep{gissler2004}. Outside the field of academic research, universal identifiers for natural persons enable work in a multi-disciplinary and multi-agency context due to greater \dfn{administrative fluency} and \dfn{bureaucratic effectiveness} \citep{alastalo2022}. This may be useful for example in the case of tackling wicked social problems that require cooperation from professionals from multiple fields, such as social workers, psychologists, police, and health care professionals.

The \pkg{hetu} and \pkg{sweidnumbr} R packages provide open tools for handling and extracting data from identification codes for natural persons and juridical persons in the context of Finland and Sweden. Prior R packages with similar scope include \CRANpkg{numbersBR} for Brazilian identity numbers for individuals, vehicles, and organisations \citep{numbersbr}, \CRANpkg{cprr} for Danish "Det Centrale Personregister" (CPR) numbers \citep{cprr}, and \CRANpkg{generator} for generating various types of Personally Identifiable Information (PII), such as fake e-mail addresses, names and the United States Social Security Numbers \citep{generator}.

Identification code generators and validators are therefore not a novel concept. At least in the case of Finland, handling Finnish identity codes seems to be a common entry-level task for new computer science students to familiarize themselves with regular expressions, dates, string subsetting and similar concepts. The reasoning behind packaging these functions as separate R packages are similar to the ones listed by \citet{wickham2022}: (1) code sharing, (2) transparency, (3) user manuals, (4) semantic versioning and (5) documenting changes between versions. In addition to these generally accepted virtues of open-source software, there is of course intrinsic value in providing a solution to a common problem in the R language.

\section{Features of identification number systems}

Reliably keeping track of individuals, organizations, objects and flows in a given territory has long been seen as an important feature of modern governance \citep{dodge2005}. All identification systems should strive to be both \dfn{unique} and \dfn{self-same} over time. Self-sameness refers to a degree of immutability that allows organizations to identify and reidentify a person over time. A combination of attributes such as name, occupation and address would probably form a unique identifier even in relatively large crowds, but such attributes might not stay the same over time. \citep{brensinger2021}.

\citet[115-120]{foucault7778} observed a historical pattern where practices first implemented in disciplinary institutions, such as prisons, military units and schools have spread to influence whole societies. We can see the results of this development today when alphanumeric identifiers originally assigned to prisoners, soldiers and students have been transformed into nationwide identification code systems around the world. Differences in legal, political and historical frameworks in different countries have affected how these systems are implemented in practice, causing heterogeneity for example in identification system designs across Europe \citep{otjacques2007}. 

This heterogeneity and linguistic differences seem to contribute to variance in the terminology used when referring to identification code systems. \dfn{Unique identifier} (UID) is an umbrella term that can be used to refer to unique identifiers for all sorts of things, from books (ISBN), chemicals (CAS) and legal entities (LEI) to anything imaginable \citep[see][]{dodge2005}. In this paper, we are mainly interested in unique identifiers for natural persons and juridical persons. 

Names such as \dfn{personal identification code} \citep{dodge2005}, \dfn{personal identity number} \citep{alastalo2022} and \dfn{single identification number} (SIN) \citep{otjacques2007} are used in literature as generic terms. Personal identification codes can sometimes be confused with \dfn{personal identification numbers} (PINs) that refer to numeric or alphanumeric passcodes used for authentication, for example, to withdraw cash or open a locked mobile phone. On the other hand names such as \dfn{personal identity code} (PIC) \citep{hetudvv, sund2012}, name number \citep{watson2010} and personal number \citep{scb2016} are used in official translations to refer to national implementations of NINs; in the mentioned cases, Finnish, Icelandic and Swedish NINs, respectively. Due to \dfn{function creep} \citep[see][]{brensinger2021, alastalo2022} or \dfn{control creep} \citep[see][]{dodge2005}, historically sector-specific identifiers may also be used as a \emph{de facto} NIN. This is the case with the US \dfn{social security number} (SSN) \citep{brensinger2021} and Finnish employee pension card numbers and social security codes \citep{alastalo2022}.

For clarity's sake, we will be using the generic term \dfn{national identification number} (NIN) to refer to all identification number systems and their country-specific implementations for natural persons, in this case, the Finnish \dfn{personal identity code} and the Swedish \dfn{personal number}. For organizations, we will use the generic term \dfn{organization identifier} when discussing Finnish \dfn{business IDs} (BID) and Swedish \dfn{organizational identity numbers} (OIN) / \dfn{Swedish organizational numbers} (SON).

According to \citet{Alterman2003}, a distinction can be made between \dfn{biocentric data} and \dfn{indexical data}. The former is biometric data connected to the individual's physical features whereas the latter has no distinguishable relation to the individual, physiologically, psychologically, or otherwise. An example of biocentric data could be a fingerprint or an iris scan and an example of indexical data could be a randomly assigned number from which nothing can be deduced.\footnote{\citet{brensinger2021} discuss the concept of \dfn{dividuals}, manufactured objects that represent the living individual: address, fingerprints, name and so on. These dividuals need to go through the process of disembedding, standardization and re-embedding to be useful. Disembedding means data gathering (taking a fingerprint sample), standardization means making the disembedded transcription into a standardized digital sample that can be easily compared with other similar samples and re-embedding means linking these standardized records back to their actual flesh-and-blood counterparts. Without a way to re-embed a huge and well-standardized archive of fingerprints back to the population, it is essentially useless. This is also a reason why biometric samples such as iris scans or fingerprints can never replace primary keys in databases.}

For several reasons, many identification numbers are not just random strings. The American SSN originally contained information about the person's birth year and where the number was first registered \citep[32]{brensinger2021} whereas Nordic countries' NINs often contain (or used to contain) information about the individual, usually birth date and sex \citep{watson2010, salste2021}. One reason for this was to make the code easier to remember \citep{alastalo2022}. Even when sex and birth date are not biocentric data in the sense as \citep{Alterman2003} defined it, including them takes Nordic NINs further away from being pure indexical data, thus making them arguably more sensitive to handle. Table 1 provides a summary of the introduction of NINs in the Nordic countries as well as information which they contain.

\begin{widetable}[ht]
\centering
\begin{tabular}{lllllll}
\toprule
  country & NIN name & introduced & characters (n) & birth date & sex & birth place \\
  \hline
  Sweden & personnummer & 1947 & 11 & yes & yes & yes\\
  Iceland & kennitala & 1950 & 10 & yes & no & no \\
  Norway & fødselsnummer & 1964 & 11 & yes & yes & no \\
  Denmark & CPR-nummer & 1968 & 11 & yes & yes & no \\
  Finland & henkilötunnus & 1968 & 11 & yes & yes & no \\
\bottomrule
\end{tabular}
\caption{Nordic NINs: year introduced and embedded information.}
\label{tab:nordiccomparison}
\end{widetable}

In the Nordic countries, comprehensive national identification number systems were developed and implemented from the 1940s to the 1960s \citep{watson2010}. In Sweden, the personal identity number (PIN) was introduced in 1947 as a way to identify individuals for tax purposes. The personal identity number consisted of both the date of birth and an additional three-digit birth number that together uniquely identified every citizen and permanent resident in Sweden. In 1967 a check digit was added to easier identify incorrect entries to finalize the full Swedish personal identity number (or \dfn{personnummer}) \citep{johansson2003,scb2016}. 

The Finnish personal identity code has its roots in specialized employment pension numbers from 1962, which was gradually expanded to cover the whole population in the form of a social security number in 1964-1968. The structure of employment pension number was designed by a Finnish mathematician Erkki Pale, who had worked in Sweden from 1948 to 1951 and was most likely inspired by the Swedish regional birth numbers \citep{alastalo2022}. His contribution was to design a more robust control character system to mitigate input errors, originally in the context of punch-card systems. The design has proved to be resilient and with some minor tweaks, it continues to be used, with the modern iteration being called a \dfn{personal identity code}\footnote{In Finnish: henkilötunnus, or \dfn{hetu} for short, hence the name of the package} \citep{salste2021}. Similar to Finland, other Nordic countries took inspiration from Sweden as well \citep{Krogness2011}. 

Table 2 illustrates the structural similarity of NINs in different Nordic countries. Some national variation does exist, however. In Norway, individual numbers are also used to denote the century in the birth date \citep{furseth2015}. In Denmark, the CPR number system was at risk of running out of unique, unused numbers and the system was amended in 2007 to turn the last check digit into a regular personal number \citep{cpr2008, jerlach2009}. The Swedish 12-digit variant omits the sometimes ambiguous century marker in favour of using a 4-digit year and it is mainly used in automatic data processing systems.

\begin{table}[ht]
\centering
\begin{tabular}{llll}
\toprule
  country & NIN name & NIN example & NIN structure \\
  \hline
  Sweden & personnummer & 610321-3499 & YYMMDDCNNNQ \\
  Iceland & kennitala & 121212-1239 & DDMMYYNNQC \\
  Norway & fødselsnummer & 01129955131 & DDMMYYCCCQQ \\
  Denmark & CPR-nummer & 300280-1178 & DDMMYY-CNNN \\
  Finland & henkilötunnus & 131052-308T & DDMMYYCNNNQ \\
  \hline
  Sweden & personnummer (ADB) & 196103213499 & YYYYMMDDNNNQ \\
\bottomrule
\end{tabular}
\caption{Examples of national identification numbers and their composition in five Nordic countries. DD: day, MM: month, YY: year, C: century marker, N: personal number numerical digit, Q: check digit or a control character.}
\label{tab:nordiccomparison2}
\end{table}

In Finland, the expansion of sector-specific social security numbers and employment pension numbers to universal NINs in 1969 has contributed to widespread secondary use of different data sources\footnote{Secondary data: Data that has not been collected primarily for a specific research question} in public administration, education and research and development. One example of this development is the use of the Finnish Hospital Discharge Register (FHDR), which contains data on all inpatient hospital discharges since 1967. The systematic development of record-keeping practices and staff training has made it possible to link discharge register data reliably to other data sources from 1972 onwards \citep{sund2012}. In Sweden, the NIN is currently used extensively in all parts of society, not only for taxation. It is used in education, for military service, in health care and by financial institutions and insurance companies. The role of the Swedish NIN has also made it central to register-based research \citep{scb2016}. It could be argued that the most important feature of NIN systems is the interoperability it enables between different sectors of society \citep{alastalo2022}.

\section{Working with Finnish and Swedish identification numbers}

The method of validating and extracting information from identification numbers is manually doable and simple in principle but in practice becomes unfeasible with datasets larger than a few dozen observations. The \pkg{hetu} and \pkg{sweidnumbr} packages provide easy-to-use tools for programmatic handling of Finnish and Swedish national identification numbers, Finnish Business ID codes\footnote{In Finnish: Yritys- ja Yhteisötunnus, or Y-tunnus for short} and Swedish organization numbers \footnote{In Swedish: Organisationsnummer}. As shown in Table 3, both packages share several core functions and even function names.

\begin{widetable}[ht]
\centering
\begin{tabular}{lll}
\toprule
    sweidnumbr & hetu & Description \\
  \hline
  \code{rpin} & \code{rpin} (\code{rhetu}) & Generate a vector of random NINs \\
  \code{pin\_age} & \code{pin\_age} (\code{hetu\_age}) & Calculate age from NIN \\
  \code{luhn\_algo} & \code{hetu\_control\_char} & Calculate check digit / control character from NIN \\
  \code{pin\_ctrl} & \code{pin\_ctrl} (\code{hetu\_ctrl}) & Check NIN validity \\
  \code{pin\_date} (\code{pin\_to\_date}) & \code{pin\_date} (\code{hetu\_date}) & Extract Birth date from NIN \\
  \code{pin\_sex} & \code{pin\_sex} (\code{hetu\_sex}) & Extract Sex From NIN \\
  \code{oin\_ctrl} & \code{bid\_ctrl} & Check OIN/BID validity \\
  \code{roin} & \code{rbid} & Generate a vector of random OINs/BIDs \\
  
\bottomrule
\end{tabular}
\caption{Exported functions that are shared between both \pkg{sweidnumbr} and \pkg{hetu}. Function alias in parentheses.}
\label{tab:hetu_sweidnumbr_shared_functions}
\end{widetable}

Both packages utilize R’s efficient vectorized operations, generating and validating over 5 million personal identity codes or Business Identity Codes in less than 10 minutes on a regular laptop. This can meet the practical upper limit set by the current population of Finland (5.5 million people) \citep{svt2022} and Sweden (10.5 million people) \citep{scb2022}, providing adequate headroom for the handling of relatively large registry datasets.

\subsection{The \pkg{hetu} package}

Printing a data frame containing extracted information in a structured form can be done as follows:

\begin{example}
 > library(hetu)
 > x <- c("010101A0101", "111111-111C", "290201A010M")
 > hetu(x)
\end{example}

The \code{hetu()} function is the workhorse of the \pkg{hetu} package. Without additional parameters, it prints out a data frame with all information that can be extracted from Finnish NINs as well as a single column that indicates if the NIN is valid as a whole or if it has any problems that make it invalid. For demonstration purposes the 3rd NIN listed below has an invalid date part; the 29th of February would only be a valid date if the year was a leap year, which 2001 is not.

\begin{example}
         hetu    sex p.num ctrl.char       date day month year century valid.pin
1 010101A0101 Female   010         1 2001-01-01   1     1 2001       A      TRUE
2 111111-111C   Male   111         C 1911-11-11  11    11 1911       -      TRUE
3 290201A010M Female   010         M       <NA>  29     2 2001       A     FALSE
\end{example}

The full birth year is constructed by reading the 2-digit year information and century marker; in the case of the first-row example, "01" and "A". "A" means that the person is born in the 2000s, "-" means the 1900s and "+" means the 1800s. A binary sex classification can be constructed simply by calculating if the personal number (\code{p.num} column) is an odd or an even number: Even numbers (for example "010") denote a female and odd numbers (for example "111") denote a male.

The generic way of outputting information found on individual columns is to use the standard \code{hetu()} function with extract-parameter.

\begin{example}
  > hetu("010101A0101", extract = "sex")
  [1] "Female"
  > hetu("010101A0101", extract = "date")
  [1] "2001-01-01"
\end{example}

All column names printed out by the \code{hetu()} function are valid extract parameters. Most commonly used columns have their own function wrappers that are identical in output:

\begin{example}
  > pin_sex("010101A0101")
  [1] "Female"
  > pin_date("010101A0101")
  [1] "2001-01-01"
\end{example}

By importing functions from the \CRANpkg{lubridate} package \citep{lubridate} we have also added the \code{pin\_age()} function to calculate the age of individuals in years (default), months, weeks or days. The default option is to calculate age at the current moment but it is also possible to set a specific date as a parameter at which the age is calculated.

\begin{example}
  > pin_age("010101A0101", date = "2004-02-01", timespan = "months")
  The age in months has been calculated at 2004-02-01.
  [1] 37
\end{example}

All NINs passed through the \code{hetu()} function are checked with 10 different tests to determine their validity. The results are crystallized in a single valid.pin column of the \code{hetu()} function output data frame. With the \code{hetu\_diagnostic()} function the user can print the results of these normally hidden tests.

\begin{example}
  > hetu_diagnostic("290201A010M")
  
           hetu is.temp valid.p.num valid.ctrl.char correct.ctrl.char
  1 290201A010M   FALSE        TRUE            TRUE             FALSE
    valid.date valid.day valid.month valid.year valid.length valid.century
  1      FALSE      TRUE        TRUE       TRUE         TRUE          TRUE
\end{example}

When data is inputted manually without validity checks, input errors can creep in. The control character in Finnish personal identity codes combined with validity checks in the \pkg{hetu} package can help to catch the most obvious errors. In the example above we can see that the date is incorrect, but also the control character is incorrect. We can simply try three different dates to see if the input error is in the day, month or year part, assuming that the personal number and control character parts were inputted correctly. In this manufactured example the error was in the year part, resulting in the rare leap day date being the correct one.

\begin{example}
  > example_vector <- c("290201A010M", "280201A010M", "290301A010M", "290200A010M")
  > columns <- c("valid.p.num", "valid.ctrl.char", "correct.ctrl.char", "valid.date")
  > hetu_diagnostic(example, extract = columns)
  
           hetu valid.p.num valid.ctrl.char correct.ctrl.char valid.date
  1 290201A010M        TRUE            TRUE             FALSE      FALSE
  2 280201A010M        TRUE            TRUE             FALSE       TRUE
  3 290301A010M        TRUE            TRUE             FALSE       TRUE
  4 290200A010M        TRUE            TRUE              TRUE       TRUE
\end{example}

The \pkg{hetu} package can generate a large number of personal identity codes with the \code{rpin} function. The date range of the generated identity codes can be changed with parameters, but it has a hardcoded lower limit at the year 1860 and an upper limit at the current date. It has been theorized that the oldest individuals that received a personal identity code in the 1960s were born the in 1860s. Personal identity codes are never assigned beforehand and therefore it is impossible to have valid personal identity codes that have a future date.

The function can also be used to generate so-called temporary personal identity codes. Temporary identity codes are never used as a persistent and unique identifier for a single individual but as a placeholder in institutions such as hospitals when a person does not have a Finnish NIN or the NIN is not known. They can be identified by having a personal number (p.num column in \code{hetu()} function output or NNN as in Table 2) in the range of 900-999.

Below is an example of generating 4 temporary Finnish NINs and checking their validity with the \code{pin\_ctrl} function. Since all NINs are temporary, they do not pass the check validity checks meant for normal pins if they are not explicitly allowed. A vector with no valid NINs returns a single NA.

\begin{example}
  > set.seed(125)
  > x <- rpin(n = 4, p.male = 0.25, p.temp = 1.0)
  > x
  [1] "250706-9565" "230117-940B" "291247-990E" "130271-908L"
  
  > pin_ctrl(x)
  [1] NA
  
  > pin_ctrl(x, allow.temp = TRUE)
  [1] TRUE TRUE TRUE TRUE
\end{example}

As mentioned earlier, our package also supports similarly generating and checking the validity of Finnish organization identifiers, or Finnish Business ID (BID) numbers. Despite the name, BIDs are used not only for companies and businesses but also for other types of organizations and other juridical persons. Unlike personal identity codes, BIDs do not contain any information about the company. BIDs consist of a random string of 7 numbers followed by a dash and 1 check digit, a number between 0 and 9. 

In addition, we have added support for the less known and less widely used numbering scheme for natural persons, Finnish Unique Identification (FINUID) numbers.\footnote{in Finnish: Sähköinen asiointitunniste or SATU for short} FINUID numbers are similar in the sense that they do not contain any biocentric data on the individual and are mainly used by government authorities in IT systems. FINUID numbers consist of 8 numbers and 1 control character calculated in the same way as in personal identity codes. Both of these ID numbers are an example of indexical data, as opposed to Finnish NINs which contain biocentric data in the form of birth date and sex.

\begin{example}
  > bid_ctrl(c("0000000-0", "0000001-9"))
  [1] TRUE TRUE

  > satu_ctrl("10000001N")
  [1] TRUE
\end{example}

The \pkg{hetu} package contains some functions that are not shared with the \pkg{sweidnumbr} package, the most notable being the \code{hetu()} function. These functions are listed and described in Table 4.

\begin{table}[ht]
\centering
\begin{tabular}{ll}
\toprule
    Function (alias) & Description \\
  \hline
  \code{hetu} & Finnish personal identification number extraction \\
  \code{pin\_diagnostic} (\code{hetu\_diagnostic}) & Diagnostics Tool for HETU \\
  \code{satu\_control\_char} & FINUID Number Control Character Calculator \\
  \code{satu\_ctrl} & Check FINUID Number validity \\
  
\bottomrule
\end{tabular}
\caption{Functions that are unique to the \pkg{hetu} package and have no equivalent in the \pkg{sweidnumbr} package. Function alias in parentheses.}
\label{tab:hetu_unique_functions}
\end{table}

\subsection{The \pkg{sweidnumbr} package}

The \pkg{sweidnumbr} R package has similar functionality as the \pkg{hetu} package, but for Swedish NINs and with a slightly different syntax. At the time of writing, the package has been downloaded roughly 30 000 times from CRAN\footnote{Source: CRANlogs API, data retrieved at 2022-03-22.}. The example NINs below are taken from the example published by the Swedish Tax Authority \citet{sv2007}.

\begin{example}
  > library(sweidnumbr)
  > example_pin <- c("640823-3234", "6408233234", "19640823-3230")
  > example_pin <- as.pin(example_pin)
  > example_pin

  [1] "196408233234" "196408233234" "196408233230"
  Personal identity number(s)
\end{example}

Unlike the \pkg{hetu} package, the \pkg{sweidnumbr} takes advantage of a custom S3 class structure. Therefore the first step is to convert strings with different Swedish NIN formats or numeric variables into a \code{pin} vector using the \code{as.pin()} function. The \code{as.pin()} function formats all inputted numbers to a so-called ADB-format\footnote{In Swedish: \dfn{automatisk databehandling}, meaning \dfn{automatic data processing} (ADP) in English} with 12 digits and no century marker, which results in less ambiguity and no need to change the century marker from "-" to "+" when a person turns 100 years old. The \code{pin} vector is an S3 object and can be checked by using the \code{is.pin()} function.

\begin{example}
  > is.pin(example_pin)

  [1] TRUE
\end{example}

This function only checks that the vector is a \code{pin} object, but not if the actual NINs are valid. To check the Swedish NIN using the control numbers, or check digits, we simply use the \code{pin\_ctrl()} function.

\begin{example}
  > pin_ctrl(example_pin)

  [1]  TRUE  TRUE FALSE
\end{example}

Just as in the \pkg{hetu} package we can extract information from the Swedish NIN with specialized functions. We can now use \code{pin\_birthplace()}, \code{pin\_sex()}, and \code{pin\_age()} to extract information on county of birth (for NINs assigned before 1990), sex, and age.

\begin{example}
  > pin_sex(example_pin)
   
  [1] Male Male Male
  Levels: Male

  > pin_birthplace(example_pin)
   
  [1] Gotlands län Gotlands län Gotlands län
  28 Levels: Stockholm stad Stockholms län Uppsala län ... Born after 31 december 1989

  > pin_age(example_pin)
  [1] 55 55 55
  > pin_age(example_pin, date = "2000-01-01")
  [1] 35 35 35
\end{example}

As with the \pkg{hetu} R package, we can also generate, or simulate, NINs with the \code{rpin()} function. Shared functions exist also for Swedish organization identifiers, or Swedish organizational numbers (SON), in the form of \code{as.oin()}, \code{is.oin()}, and \code{oin\_ctrl()} functions. Unlike the Finnish BID, the \code{oin} number contains information on the type of organization of a given SON, which can be determined by using the \code{oin\_group()} function.

\begin{example}
  example_oin <- c("556000-4615", "232100-0156", "802002-4280")
  > oin_group(example_oin)

  [1] Aktiebolag
  [2] Stat, landsting, kommuner, församlingar
  [3] Ideella föreningar och stiftelser
  3 Levels: Aktiebolag ... Stat, landsting, kommuner, församlingar
\end{example}

Similar to the \code{rbid()} function from the \pkg{hetu} package, we can generate new SONs using the \code{roin()} function from the \pkg{sweidnumbr} package.

\begin{example}
  > set.seed(125)
  > roin(3)

  [1] "776264-6144" "274657-0148" "827230-7631"
  Organizational identity number(s)
\end{example}

Due to the national characteristics of Swedish numbering schemes for natural and juridical persons some functions are unique to the \pkg{sweidnumbr} packages. These functions are listed in Table 5.

\begin{table}[ht]
\centering
\begin{tabular}{ll}
\toprule
    Function & Description \\
  \hline
  \code{as.oin} & Parse organizational identity numbers \\
  \code{as.pin} & Parse personal identity numbers to ADP format \\
  \code{format\_pin} & Formatting pin \\
  \code{is.oin} & Test if a character vector contains correct 'oin' \\
  \code{is.pin} & Parse personal identity numbers to ADP format \\
  \code{oin\_group} & Calculate organization group from 'oin' \\
  \code{pin\_birthplace} & Calculate the birthplace of 'pin' \\
  \code{pin\_coordn} & Check if 'pin' is a coordination number \\
\bottomrule
\end{tabular}
\caption{Functions that are unique to the \pkg{sweidnumbr} package and have no equivalent in the \pkg{hetu} package.}
\label{tab:hetudiagnostics}
\end{table}

\section{Discussion}

The \pkg{hetu} and \pkg{sweidnumbr} R packages provide free and open-source methods for validating and extracting data from a large number of Finnish and Swedish national identity numbers (NIN). While the packages' target audience most likely mainly consists of Finnish and Swedish users and people with a particular interest in NIN systems around the world, the packages make a generic contribution to developing methodologies related to NIN handling in R, and more generally for {\it structured data} in the field of computational humanities \citep[see][]{makela2020}, epidemiology and demographic studies \citep[see][]{gissler2004}. A possible direction for future developments could be to create more generic class structures or even a completely new R package that could recognize and handle NIN systems from several different countries around the world.

The origins of the \pkg{hetu} package can be traced to the early 2010s when one curious individual wanted to analyze a large number of Finnish NINs that were leaked to the internet by an anonymous hacker, to identify the source of the leak. The legality and morality of handling such datasets containing personal information was and is in a grey area at best. As developers of these packages, we cannot condone such activities, even if they are conducted out of curiosity and not of ill intentions, but we acknowledge that we cannot prevent our users from doing that either. Both \pkg{hetu} and \pkg{sweidnumbr} packages are free software with permissive licenses and pre-emptively limiting their use to only "good, not evil" causes would be problematic as well.\footnote{For example, JSON has a license that states that "The Software shall be used for Good, not Evil". Defining what is good and evil is at least in part up to everyone's personal judgment, making the license clause ambiguous.}

We have acknowledged beforehand that random NINs generated with the \pkg{hetu} and \pkg{sweidnumbr} packages could, theoretically, be used for purposes such as synthetic identity fraud.\footnote{see \citet[32]{brensinger2021} for a short description of synthetic fraud related to American SSNs} On the other hand it is important to note that such NINs could also be created by hand as information on valid NINs is readily available e.g. on the Finnish Digital and Population Data Services Agency and Swedish Tax Authority websites \citep{hetudvv, sv2007}. Our package can be useful for many, and it does not make fraudulent activities significantly easier for malevolent individuals, which is essential in judging the pros and cons of releasing this software to the public.

Similar data breaches have made people warier about digital services. Privacy concerns can push Finland, Sweden and other Nordic countries towards redesigning their national identification numbers to omit some or all of the embedded personal information sometime in the future. For example, at the time of writing there is an ongoing government project led by the Finnish Ministry of Finance to redesign the Finnish NIN structure \citep{valtiovarainministerio2022}. We will continue to monitor for such policy changes and make changes to the packages if necessary.

As mentioned earlier, both packages are published under a permissive BSD 2-clause license. We encourage our users to give feedback on the packages and their materials, report bugs or any legislative or policy changes related to NIN system implementations, study the source code and submit improvements to our public code repositories or fork the code to better suit their needs.\footnote{\url{https://github.com/rOpenGov/hetu}, \url{https://github.com/rOpenGov/sweidnumbr}}

\section{Acknowledgements}

We are grateful to all contributors, in particular Juuso Parkkinen and Joona Lehtomäki for their support in the initial package development. This work is part of rOpenGov\footnote{\url{http://ropengov.org}} and contributes to the FIN-CLARIAH research infrastructure for computational humanities. LL and PK were supported by the Academy of Finland (decisions 295741, 345630).

\bibliography{lahti}

\address{Pyry Kantanen\\
  Department of Computing\\
  PO Box 20014 University of Turku\\
  Finland\\
  ORCiD: 0000-0003-2853-2765\\
  \email{pyry.kantanen@utu.fi}}
  
\address{Erik Bülow\\
  Department of Orthopaedics, Institute of clinical sciences\\
  Sahlgrenska Academy\\ University of Gothenburg\\
  Sweden\\
  ORCiD: 0000-0002-9973-456X\\
  \email{erik.bulow@gu.se}}
  
\address{Måns Magnusson\\
  Department of Statistics\\
  Uppsala University\\
  Sweden\\
  ORCiD: 0000-0002-0296-2719\\
  \email{mans.magnusson@statistik.uu.se}}
  
\address{Jussi Paananen\\
  Institute of Biomedicine\\
  University of Eastern Finland\\
  Finland\\
  ORCiD: 0000-0001-5100-4907\\
  \email{jussi.paananen@uef.fi}}

\address{Leo Lahti\\
  Department of Computing\\
  PO Box 20014 University of Turku\\
  Finland\\
  ORCiD: 0000-0001-5537-637X\\
  \email{leo.lahti@utu.fi}}
